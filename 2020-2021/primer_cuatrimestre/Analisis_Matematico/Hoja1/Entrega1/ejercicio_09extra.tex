\documentclass[12pt]{article}

%\input xy
%\xyoption{all}

\DeclareMathAlphabet{\mathpzc}{OT1}{pzc}{m}{it}

\usepackage{amsfonts,amsmath,amssymb,indentfirst,mathrsfs,amscd}
\usepackage[mathscr]{eucal}
%\usepackage[active]{srcltx}
\usepackage{graphicx}
\usepackage{nicefrac}
\usepackage{cases}



\usepackage[usenames,dvipsnames]{xcolor}
\usepackage[utf8]{inputenc}
%\usepackage[spanish]{babel}
%\usepackage[clock]{ifsym}


\newcommand{\CC}{\mathbb{C}} %Complex numbers%
\newcommand{\QQ}{\mathbb{Q}} %Rational numbers%
\newcommand{\PP}{\mathbb{P}} %Projective space%
\newcommand{\HH}{\mathbb{H}} %Hypercohomology, quaternions..%
\newcommand{\RR}{\mathbb{R}} %Real numbers%
\newcommand{\ZZ}{\mathbb{Z}} %Integer numbers%
\newcommand{\NN}{\mathbb{N}} %Natural numbers%

\usepackage{enumerate}

\pagestyle{empty}
\oddsidemargin 0pt
\topmargin -2.5cm
\evensidemargin 0pt
\textwidth 16.5cm
\textheight 24cm

%%%%%%%%%%%%%%%%%%%%%%%%%%%%%%%%%%%
%%%%%%%%%%%%%%%%%%%%%%%%%%%%%%%%%%%
%%%%%%%%%%%%%%%%%%%%%%%%%%%%%%%%%%%
%%%%%%%%%%%%%%%%%%%%%%%%%%%%%%%%%%%


\begin{document}
\hrule\hrule
\vspace{2mm}

\noindent {\bf Análisis Matemático, curso 2020-21 \hfill{Matemáticas-Ingeniería informática}}

\vspace{3mm}

 \noindent {\bf Ejercicio adicional}: debilitando hipótesis\hfill {\it 
 Nombre: Junco de las Heras Valenzuela}

\vspace{2mm}

\hrule\hrule

\vspace{2mm}

\

\noindent  {\bf Problema 9 extra.} Sea $D:X\times X\to \RR$ una función tal que:
\begin{itemize}
	\item $D(x,y)=0 \iff x = y$.
	\item $D(x,y) \leq D(z,x) + D(z,y)$ para todo $x,y,z$.
\end{itemize}
Demostrar que $D$ es una distancia en $X$.

\noindent {\bf{\underline{Observación}: }}Aquí, a diferencia de en la hoja de problemas, no se está asumiendo que los valores de $D$ sean no negativos.

\vspace{2mm}
\hrule
\vspace{2mm}


\noindent Sustituyendo en la segunda ecuación $z$ por $y$ obtenemos:
\[
D(x, y) \leq D(y, x) + D(y, y)
\] 
Usando que $D(y, y)$ = 0:
\[
D(x, y) \leq D(y, x)
\]
Análogamente tomando $x$ como $y$, $y$ como $x$ y $z$ como $x$ obtenemos:
\[D(y, x) \leq D(x, x) + D(x, y)
\]
Usando que $D(x, x) = 0$:
\[
D(y, x) \leq D(x, y)
\]
Uniendo las desigualdades:
\[
D(x, y) \leq D(y, x) \leq D(x, y)
\]
Implica que $D(x, y) = D(y, x)$, y $D$ es simétrica.\\\\
Tomando $y$ como $x$, y $z$ como $y$ queda:
\[
D(x, x) \leq D(y, x) + D(y, x)
\]
$D(x, x) = 0$, y $D$ es simétrica:

\[
\begin{aligned}
	0 &\leq 2*D(x, y) \\
	0 &\leq D(x, y)	
\end{aligned}
\]
Cumpliendo todas las propiedades para que D sea una función distancia.

 
 \end{document}
%%%%%%%%%%%%%%%%%%%%%%%%%%%%%%%%%%%
%%%%%%%%%%%%%%%%%%%%%%%%%%%%%%%%%%%
%%%%%%%%%%%%%%%%%%%%%%%%%%%%%%%%%%%
%%%%%%%%%%%%%%%%%%%%%%%%%%%%%%%%%%%

